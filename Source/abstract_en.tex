\begin{abstract}
	The development of the field of machine learning has allowed us to solve problems unattainable by traditional research methods up to this point.
	However, the quality of such learning often depends on the data used in the process.
	Data quality is a~topic that arises fairly often and is widely regarded as an essential element of data processing in general, but there is no clear and well-researched way of describing it or determining what that quality really is.
	The syntactic characteristics of datasets that can allow data quality analysis from a~practical side of view can be captured and represented in the form of metadata.
	Previous research has analysed metadata in the context of problems related to it, the applicability of categorization algorithms and attempts to state the ontology of dataset quality.
	In this work, I~undertake a~meta-analysis examining the relationship between metadata describing characteristics of collections and their utility and functional attractiveness.
	I am investigating the relevant correlations and trying to figure out whether there are links between belonging to certain syntactic categories and the popularity of the collection and what those regularities are.
	I also suggest how to test and refine my results in future studies.
\end{abstract}
