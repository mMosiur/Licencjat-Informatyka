\chapter{Metaanaliza i analiza metadanych} \label{ch:chapter1}

Metaanaliza oznacza wtórne odkrywanie wiedzy za pomocą uogólnienia informacji znajdujących się w pierwotnych źródłach \cite{higgins2019cochrane}.
W przypadku tej pracy istotny jest aspekt metaanalizy polegający na \textit{analizie metadanych}, czyli przeprowadzeniu analizy z użyciem standardowych narzędzi, ale na metadanych zbioru zamiast danych właściwych.
W rezultacie warto najpierw przyjrzeć się samej analizie danych.

\section{Analiza danych} \label{sec:1AnalizaDanych}

Otacza nas niemal nieskończony zasób informacji o świecie.
Nawet biorąc pod uwagę jedynie dane cyfrowe IDC (\textit{International Data Corporation}) estymowała, że rocznie w roku 2020 wytwarzanych będzie 35 zetabajtów danych \cite{tien2013big}.
Ilość ta została osiągnięta już dwa lata wcześniej; W 2018 pojawiały się wzmianki mówiące, że już wtedy roczna ilość wytwarzanych danych osiągnęła 33 zetabajty \cite{Patrizio:2018}, a aktualnie IDC mówi o nawet 59 zetabajtach \cite{IDC:2020}.
Mimo, że generowane są tak potężne zasoby danych, aby móc je wykorzystać trzeba do nich dotrzeć oraz odpowiednio przygotować.
Bez tego ich analiza staje się niezwykle trudna lub niemal niemożliwa.

	\subsection{Zdobywanie danych} \label{subsec:1ZdobywanieDanych}
	Sama obecność informacji w środowisku nie jest wystarczająca aby móc je w jakikolwiek sposób wykorzystać.
	Proces \textit{Data Science} składa się z pięciu kroków, z których pierwszym jest właśnie wydobycie danych.

	W nauce eksperymentalnej, w tym nawet w jej najświeższych gałęziach na początku fazy opisu jakościowego, badania zjawisk oparte są na pomiarach \cite{brandt1998data}.
	Dzisiaj wszelkie pomiary, zarówno w nauce jak zastosowaniach praktycznych, są w zdecydowanej większości rezultatem działania sensorów.
	Struktura i sposób ich działania może być zróżnicowany i złożony \cite{deshpande2004model,boyer2009scada}, ale cel jest jeden - przechwycić interesujące nas dane o świecie zewnętrznym.
	Działanie takich sensorów niestety często generuje dane nieustrukturyzowane, które sprawiają duży problem w zarządzaniu i są trudne do efektywnego przeanalizowania \cite{blumberg2003problem}.
	



	\subsection{Przetwarzanie danych} \label{subsec:1PrzetwarzanieDanych}
	\subsection{Analiza danych} \label{subsec:1AnalizaDanych}

\section{Analiza metadanych} \label{sec:1AnalizaMetadanych}
	\subsection{Metadane} \label{subsec:1Metadane}