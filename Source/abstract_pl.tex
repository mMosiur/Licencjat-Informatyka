\begin{abstract}
	Rozwój dziedziny uczenia maszynowego pozwolił na rozwiązywanie problemów dotychczas nieosiągalnych tradycyjnym metodom badawczym.
	Jakość takiego uczenia często jednak zależy od danych użytych w~procesie.
	Jakość danych jest tematem, który się często pojawia i~jest powszechnie uważana za istotny element ogólnie pojętego przetwarzania danych, jednak nie ma wyprowadzonego i~dobrze zbadanego sposobu jej określenia czy powiedzenia, na czym ta jakość tak na prawdę polega.
	Syntaktyczne cechy zbiorów danych, które mogą pozwolić na analizę jakości danych z~praktycznej strony, mogą być uchwycone i~reprezentowane w~postaci metadanych.
	Dotychczasowe prace badawcze analizowały metadane w~kontekście problemów z~nimi związanych, stosowalności algorytmów kategoryzujących oraz próby określenia ontologii jakości zbiorów danych.
	W tej pracy podejmuję metaanalizę badając zależności między metadanymi dotyczącymi cech zbiorów a~ich użytecznością i~atrakcyjnością użytkową.
	Badam stosowne korelacje oraz próbuję odnaleźć, czy istnieją powiązania między przynależnością do pewnych kategorii syntaktycznych a~popularnością zbioru i~jakie są to prawidłowości.
	Sugeruję również sposób, w~jaki można przetestować i~doprecyzować rezultaty mojego badania w~przyszłych pracach.
\end{abstract}
