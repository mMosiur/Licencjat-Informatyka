\begin{abstract}
	Analiza danych o otaczającym nas świecie jest jedną z kluczowych metod postępu, zarówno naukowego jak i technologicznego.
	Skuteczność takiej analizy jest jednak zależna od jakości dostępnych informacji.
	Otacza nas niemal nieskończony zasób danych o świecie, jednak aby móc go wykorzystać trzeba do nich dotrzeć, zdobyć oraz odpowiednio przygotować.
	W wyniku takiej operacji powstaje baza danych, jednak jej jakość nie zawsze jest jednakowa;
	dane nieustrukturyzowane będą trudniejsze do przeanalizowania niż dane ustrukturyzowane.
	W tej pracy zamierzam wykonać uproszczoną metaanalizę cech baz danych sprawdzając, które z nich wpływają na ich użyteczność i popularność.
\end{abstract}