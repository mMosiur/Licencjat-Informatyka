% Analiza danych o otaczającym nas świecie jest jedną z kluczowych metod postępu, zarówno naukowego jak i technologicznego.
% Skuteczność takiej analizy jest jednak zależna od jakości dostępnych danych.

% Problem z nieustrukturyzowanymi danymi zauważalny był jeszcze długo zanim ilości generowanych danych sięgały ułamka dzisiejszego natłoku informacji \cite{blumberg2003problem}, a wraz z czasem problem ten się tylko nasila.
% Aby móc skutecznie wnioskować z użyciem danego zbioru informacji powinien on być ustrukturyzowany.
% Najpowszechniejszym przykładem ustrukturyzowania są zbiory danych (\textit{datasets}), najczęściej w postaci tabelarycznej.
% Często używane jest pojęcie \textit{baz danych} zamiennie ze zbiorami danych, jednak nie jest to prawidłowe.
% Bazy danych to kontenery zawierające zarówno zbiory danych, jak i całą infrastrukturę służącą do interakcji i manipulacji danymi znajdującymi się wewnątrz.
% Zawartych zbiorów często jest więcej niż jeden i są powiązane relacyjnie między sobą.
% Istnieją archiwa przechowujące tego typu ustrukturyzowane zbiory danych i udostępniające je publicznie do wykorzystania najczęściej do celów badawczych.
% Jedno z nich to UC Irvin Machine Learning Repository (\url{https://archive.ics.uci.edu/}), gdzie poza samymi zbiorami danych znajdują się na ich metadane.

% Metadane zbioru danych to cechy opisujące sam zbiór, takie jak ilość rekordów (krotek, wierszy) czy atrybutów (kolumn).
% Oczywistym jest, że w użyteczności danego zbioru najistotniejsza jest jego semantyka, informacje w nim zawarte.
% Jednak syntaktyka również odgrywa swoją rolę; jednym z bardziej oczywistych przykładów może być ilość danych.
% W zdecydowanej większości im więcej danych zawartych jest w zbiorze tym lepiej, chociaż i od tej reguły pojawiają się wyjątki \cite{gasco2012does,boivin2006more}.
% Na jakość zbioru wpływ ma także obecność brakujących danych, mimo istnienia algorytmów pozwalających na ich uzupełnienie \cite{acuna2004treatment}.

% W tej pracy przeanalizuję metadane zbiorów danych i ich wpływ na popularność i użyteczność zbioru.
% Przedstawię również projekt badawczy analogiczny do mojego pozwalający na dokładniejszą i rozszerzoną metaanalizę, aby docelowo odnaleźć zależności syntaktyczne i pomóc w tworzeniu zbiorów danych bogatszych i łatwiejszych w analizie.

\chapter{Wstęp}
\label{ch:introduction}

Uczenie maszynowe jest fantastycznym narzędziem pozwalającym na wykonywanie zadań dotychczas niedostępnych tradycyjnym algorytmom poprzez wykorzystanie kluczowej zdolności uczenia się i samo-ulepszenia.
Trzema najważniejszymi jego zastosowaniami są programy samo-przystosowujące, aplikacje, których nie można zaprogramować ręcznie ze względu na ich złożoność oraz użycie danych historycznych w celu ulepszenia przewidywania i podejmowania decyzji \cite{mitchell1997machine}.
Żadne z tych zastosowań nie byłoby jednak możliwe bez dostępności odpowiednich danych.

Dane i zbiory danych są kluczowym elementem systemów uczenia maszynowego.
Potrzebne w konkretnym zastosowaniu dane można wygenerować lub zebrać osobiście, jednak istnieją dedykowane repozytoria pozwalające na wyszukanie i wykorzystanie interesujących zbiorów danych oraz dotację tych przez nas zebranych.
Taka ogólnodostępność pozwala to na wielokrotne użycie tych samych danych w różnych zastosowaniach lub przez różne osoby.
Jako przykład można podać znany wielu osobom zbiór odręcznie zapisanych cyfr MNIST.
Autorzy zbioru zadbali o jego jakość, znormalizowali wielkość obecnych tam cyfr, zastosowali antyaliasing oraz wycentrowali powstałe obrazki względem ich środka ciężkości \cite{mnist}.
Zabieg ten był niezwykle istotny, ponieważ pośrednio wpłynął na jakość wszystkich badań i analiz prowadzonych z użyciem tego zbioru danych.

Obrazuje to wagę, jaką powinno się przykładać do utrzymania jakości zbiorów danych.
Ma tutaj bowiem zastosowanie popularna w informatyce zasada \textit{śmieci na wejściu - śmieci na wyjściu} (ang. \textit{"garbage in - garbage out"}).
Mówi ona, że nawet w przypadku idealnej logiki programu, podanie na wejściu niepoprawnych danych wygeneruje niepoprawne wyniki.
W tym kontekście uczenie systemu z wykorzystaniem danych o niskiej jakości spowoduje, że rezultaty generowane przez ten system również będą niskiej jakości czy nieprecyzyjne.

Przez taką jakość można rozumieć semantyczną zawartość zbioru, gdzie obecność błędnych danych w sposób oczywisty wpływa na działanie systemu.
Jednak często pomijana jest \textit{syntaktyka zbioru danych}, czyli sposób ustrukturyzowania czy reprezentacji zawartych danych.
Zmiany wprowadzone do zbioru MNIST przez jego autorów można,przynajmniej częściowo, uznać za zmiany tego właśnie typu.
Ilość zawartych danych również jest wartością syntaktyczną i w sposób oczywisty wpływa na jakość uczenia.

W kontekście zbiorów danych ich syntaktyka jest reprezentowana przez \textit{metadane}.
Zależności i wpływ metadanych na jakość i użyteczność zbiorów danych nie był dotychczas dobrze przebadany.
Dotychczas pojawiały się podobne metaanalizy, jednak w innych kontekstach, takich jak aplikowalność algorytmów \cite{brazdil1994characterizing}.

W tej pracy przeanalizuję metadane w ręcznie zebranej bazie zbiorów danych, określając jednocześnie projekt przeprowadzenia w przyszłych pracach podobnej metaanalizy dla danych z poprawionymi metrykami.
