\chapter*{Wstęp}
\addcontentsline{toc}{chapter}{Wstęp}
\label{ch:introduction}

Uczenie maszynowe jest fantastycznym narzędziem pozwalającym na wykonywanie zadań dotychczas niedostępnych tradycyjnym algorytmom poprzez wykorzystanie kluczowej zdolności uczenia się i~samo-ulepszenia.
Trzema najważniejszymi jego zastosowaniami są programy samo-przystosowujące, aplikacje, których nie można zaprogramować ręcznie ze względu na ich złożoność oraz użycie danych historycznych w~celu ulepszenia przewidywania i~podejmowania decyzji \cite{mitchell1997machine}.
Żadne z~tych zastosowań nie byłoby jednak możliwe bez dostępności odpowiednich danych.

Dane i~zbiory danych są kluczowym elementem systemów uczenia maszynowego.
Potrzebne w~konkretnym zastosowaniu dane można wygenerować lub zebrać osobiście, jednak istnieją dedykowane repozytoria pozwalające na wyszukanie i~wykorzystanie interesujących zbiorów danych oraz dotację tych przez nas zebranych.
Taka ogólnodostępność pozwala na wielokrotne użycie tych samych danych w~różnych zastosowaniach lub przez różne osoby.
Jako przykład można podać znany wielu osobom zbiór odręcznie zapisanych cyfr MNIST \cite{mnist}.
Autorzy zbioru zadbali o~jego jakość, znormalizowali wielkość obecnych tam cyfr, zastosowali antyaliasing oraz wycentrowali powstałe obrazki względem ich środka ciężkości.
Zabieg ten był niezwykle istotny, ponieważ pośrednio wpłynął na jakość wszystkich badań i~analiz prowadzonych z~użyciem tego zbioru danych.

Obrazuje to wagę, jaką powinno się przykładać do utrzymania jakości zbiorów danych.
Ma tutaj bowiem zastosowanie popularna w~informatyce zasada \textit{śmieci na wejściu -- śmieci na wyjściu} (ang. \textit{``garbage in -- garbage out''}).
Mówi ona, że nawet w~przypadku idealnej logiki programu, podanie na wejściu niepoprawnych danych wygeneruje niepoprawne wyniki.
W tym kontekście uczenie systemu z~wykorzystaniem danych o~niskiej jakości spowoduje, że rezultaty generowane przez ten system również będą niskiej jakości, nieprecyzyjne czy po prostu niepoprawne.

Przez taką jakość można rozumieć semantyczną zawartość zbioru, gdzie obecność błędnych danych w~sposób oczywisty wpływa na działanie systemu.
Jednak często pomijana jest \textit{syntaktyka zbioru danych}, czyli sposób ustrukturyzowania czy reprezentacji zawartych danych.
Zmiany wprowadzone do zbioru MNIST przez jego autorów można, przynajmniej częściowo, uznać za zmiany tego właśnie typu.
Powstaje pytanie jakie z~tych wartości syntaktycznych mają wpływ na jakość zbioru oraz w~jaki sposób na niego oddziałują.
Niektóre z~nich, takie jak właśnie ilość zawartych rekordów czy obecność brakujących danych wydają się mieć oczywisty wpływ.
Można się jednak zastanowić nad innymi, jak rodzaj zawartych danych czy ilość atrybutów.
W kontekście zbiorów danych ich syntaktyka może być reprezentowane przez ich \textit{metadane}.
Jeśli zbierzemy wszystkie te cechy w~postaci metadanych, można przeprowadzić ich analizę w~celu znalezienia zależności i~powiązań.

Zależności między metadanymi a~jakością i~użytecznością ich zbiorów danych nie były dotychczas dobrze przebadane.
Obecnie jedyne dostępne metaanalizy opierające się na metadanych zbiorów, które zostały przeprowadzone, opierają się na innych kontekstach, takich jak stosowalność algorytmów uczenia maszynowego \cite{brazdil1994characterizing}.

W tej pracy przeanalizuję metadane reprezentujące cechy i~własności zbiorów danych zawartych w~UC Irvin Machine Learning Repository \cite{Dua:2021}.
Dodatkowo przedstawię sposób pozyskania danych i~skonstruowania zbioru przy braku publicznie dostępnego API oraz sposób analizy takich danych.
Pozyskane przeze mnie dane oraz program Scrapera, który do tego został użyty dostępny jest jako załącznik do tej pracy lub pod adresem \url{https://github.com/mMosiur/morus2021metaanalysis}.
